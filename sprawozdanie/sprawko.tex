\documentclass[a4paper,10pt]{article}
\usepackage[top=0.85in,right=1in,left=1in,bottom=0.85in]{geometry}
\usepackage{latexsym}
\usepackage{amsfonts}
\usepackage{amssymb}
\usepackage{amsmath}
\usepackage{amsthm}

\usepackage{mathspec}
\usepackage{color}
\usepackage[english,polish]{babel}
\usepackage[plmath]{polski}
\usepackage{xltxtra}
\usepackage{icomma}
\usepackage{textcomp}
\usepackage{multirow}

\usepackage{subfig}
\usepackage{float}
\usepackage{pdfpages}

%\usepackage{minted}

\usepackage{graphicx}

\usepackage{booktabs}
\usepackage[exponent-product = \cdot, per-mode = fraction,%
binary-units, output-decimal-marker = \mbox{,}]{siunitx}

\addto\captionspolish{%
  \renewcommand\tablename{Tabela}
}

% Zmiana rozmiarów strony tekstu
\addtolength{\voffset}{-1cm}
\addtolength{\hoffset}{-1cm}
\addtolength{\textwidth}{2cm}
\addtolength{\textheight}{2cm}

%bardziej zyciowe parametry sterujace rozmieszczeniem rysunkow
\renewcommand{\topfraction}{.85}
\renewcommand{\bottomfraction}{.7}
\renewcommand{\textfraction}{.15}
\renewcommand{\floatpagefraction}{.66}
\renewcommand{\dbltopfraction}{.66}
\renewcommand{\dblfloatpagefraction}{.66}
\setcounter{topnumber}{9}
\setcounter{bottomnumber}{9}
\setcounter{totalnumber}{20}
\setcounter{dbltopnumber}{9}

% w³asny bullet list z malymi odstepami
\newenvironment{tightlist}{
\begin{itemize}
  \setlength{\itemsep}{1pt}
  \setlength{\parskip}{0pt}
  \setlength{\parsep}{0pt}}
{\end{itemize}}

%obrazkow szukamy w nastepujacym katalogu:
\graphicspath{{pics/}}



\begin{document}

\thispagestyle{empty} %bez numeru strony

\begin{center}
{\large{Sprawozdanie z laboratorium:\\
Metaheurystyki i Obliczenia Inspirowane Biologicznie\\}}

\vspace{3ex}

Część I: Algorytmy optymalizacji lokalnej, problem QAP
%Część II: Algorytmy optymalizacji lokalnej i globalnej, problem QAP
%Część III: Eksperyment: ... (prezentację można zrobić w LaTeX - służy do tego klasa "beamer")

\vspace{3ex}
{\footnotesize\today}

\end{center}


\vspace{10ex}

Prowadzący: dr hab.~inż. Maciej Komosiński

\vspace{5ex}

Autorzy:
\begin{tabular}{lllr}
\textbf{Jacek Jankowski} & inf99410 & ISWD & jacek.j.jankowski@gmail.com \\
\textbf{Benedykt Jaworski} & inf99893 & ISWD & sth.jaworski@gmail.com \\
\end{tabular}

\vspace{5ex}

Zajęcia poniedziałkowe, 15:10.

\vspace{35ex}

\noindent Oświadczamy, że niniejsze sprawozdanie zostało przygotowane wyłącznie przez powyższych autorów,
a~wszystkie elementy pochodzące z~innych źródeł zostały odpowiednio zaznaczone i~są cytowane w~bibliografii.  

\newpage



\section{Quadratic Assignment Problem}
\subsection{Opis problemu}
QAP jest problemem kombinatorycznym NP-trudnym. 
Nie istnieją algorytmy pozwalające na rozwiązanie tego problemu w czasie wielomianowym.
QAP jest matematycznym modelem dla zadania umieszczenia połączonych ze sobą elementów na płytce drukowanej (elektronika) lub wyborze lokalizacji jednostek firmy (administracja).
Problem można opisać w następujący sposób:
%TODO Wcięcie poniższego bloku tekstu (jak definicja w książce)
Istnieje n jednostek i n lokalizacji.
Dla każdej pary jednostek istnieje \emph{dystans}, a dla każdej pary lokalizacji istnieje \emph{przepływ} (nazywany również wagą).
Przepływ może być np. ilością materiałów transportowanych pomiędzy magazynem a halą produkcyjną lub szerokością ścieżki na płytce drukowanej.
Zdanie polega na przydzieleniu jednostkom lokalizacji w taki sposób, aby zminimalizować sumę iloczynów dystansów pomiędzy jednostkami i przepływów pomiędzy nimi.
% --- Koniec wcięcia ---

Formalna definicja problemu QAP jest następująca:
Dane są dwa zbiory, $P$ ("jednostki") i $L$ ("lokalizacje"), równych rozmiarów, a także funkcja wagi 
%TODO Pogrubić R
$w: P \times P \rightarrow R$ i funkcja odległości $d: L \times L \rightarrow R$.
Znajdź funkcję przypisującą jednostkom lokalizacje $f: P \rightarrow L$ w taki sposób, że funkcja kosztu:
%TODO znak zawiera się po a,b
$$\sum_{a,bP}w(a,b) \cdot d(f(a),f(b))$$
jest najmniejsza możliwa.

Tak zdefiniowana funkcja kosztu intuicyjnie zachęca do umieszczania pozycji o dużym przepływie w niewielkiej odległości.

\subsection{Reprezentacja i operator sąsiedztwa}
Rozwiązaniem problemu jest permutacja jednostek.
Kolejność jednostek w permutacji określa ich położenie i pozwala odczytać odległości z macierzy odległości.
Użyty operator sąsiedztwa to sąsiedztwo 2-opt, zamieniające miejscami dwa elementy w permutacji.

Złożoność obliczeniowa uzyskania wartości kosztu aktualnej permutacji ma charakter $O(n^2)$, gdzie n to rozmiar permutacji.
Zaimplementowaliśmy funkcję, która sprawdza o ile zmieni się koszt, jeżeli użyjemy operatora sąsiedztwa na danej permutacji.
Funkcja ta ma złożoność liniową, co pozwoliło przyspieszyć działanie algorytmu.
%TODO do ilu?
Gdyby uwzględnić fakt, że większość problemów jest symetryczna, można by zmniejszyć liczbę wymaganych obliczeń.

Rozmiar sąsiedztwa dla operatora 2-opt to $\frac{n \cdot (n-1)}{2}$. 

\subsection{Algorytmy}
Ziarno generatora liczb losowych jest ustawiane na początku działania programu.
Wartość jest taka sama dla każdego uruchomienia, co gwarantuje powtarzalność eksperymentów.
%TODO Opisać generator?
Liczba uruchomień algorytmu losowego była proporcjonalna do rozmiary problemu.
Na podstawie czasu wykresu czasu działania stwierdziliśmy, że gwarantuje to czas działania podobny do czasów działania algorytmów Greedy i Steepest.
\subsubsection{Heuristic}
Algorytm kieruje się heurystyką i ustawia najpierw najbliżej siebie jednostki o największych przepływach.
W każdym kolejnym kroku dobierana jest największa pozostała jednostka i umieszczana na pozycji, która jest w najmniejszej odległości od już umieszczonych.
\subsubsection{Random}
Algorytm zwraca losową permutację.
\subsubsection{Greedy}
Algorytm Greedy wybiera losowe sąsiedztwo i oblicza jego koszt.
Jeżeli sąsiedztwo ma koszt niższy lub równy aktualnemu rozwiązaniu, to zastępuje ono aktualne rozwiązanie.
Jeżeli nie, algorytm sprawdza kolejne sąsiedztwo.
Jeżeli żadne sąsiedztwo nie ma kosztu równego lub niższego, algorytm kończy działanie.
Aby zapobiec zapętleniu (przechodzieniu pomiędzy dwoma permutacjami o równej ocenie) stosujemy ograniczenie do 3 przejść do sąsiedztwa o równym koszcie.
\subsubsection{Steepest descent}
Algorytm działa podobnie jak Greedy z tym, że sprawdza wszystkie sąsiedztwa i spośród nich wybiera to o najniższej wartości funkcji celu.
Aby zapobiec zapętleniu (przechodzieniu pomiędzy dwoma permutacjami o równej ocenie) stosujemy ograniczenie do 3 przejść do sąsiedztwa o równym koszcie.

\section{Eksperymenty}
\subsection{Instancje problemów}
Eksperymenty uruchamialiśmy na następujących problemach:
\begin{enumerate}
\item bur26a
\item chr12a
\item chr15a
\item chr18a
\item chr20a
\item chr22a
\item chr25a
\item had14
\item nug17
\item ecs16a
\item els19
\item esc32
\item kra30a
\item nug21
\end{enumerate}
Problemy te posiadają obliczone optimum, co ułatwiło obliczenie jakości uzyskanych przez nas rozwiązań.
%TODO Link do QAPLIB
Problemy pochodzą z repozytorium QAPLIB.
Źródłem danych są m.in. czas pisania stenotypisty, macierze sąsiedztwa drzew czy testy układów elektronicznych.

Algorytmy Greedy oraz Steepest descent zostały uruchomione 200 razy, a wyniki i parametry działania zagregowane. 
Algorytm losowy został uruchomiony tyle razy, by czas jego działania odpowiadał rządem wielkości czasowi działania algorytmów Greedy i Steepest descent.
Poniższe wykresy przedstawiają anlizę pracy i sprawności algorytmów.
\subsection{Jakość działania jako funkcja rozmiaru problemu}
\subsubsection{Średnia jakość rozwiązania}
Jakość działania liczona jest jako stosunek kosztu rozwiązania znalezionego do kosztu najlepszego rozwiązania znalezionego przez algorytm.
Poniższy wykres przedstawia wartości średnie jakości (punkty) oraz wartość 1 odchylenia standardowego (linie pionowe).
\begin{center}
\input{./plotQualitySizeMean.tex}
\end{center}
%TODO Jaki separator części dziesiętnej stosujemy?
Z wykresu odczytujemy, że średnia jakość działania algorytmu Random jest dla większości problemów niska i nie przekracza wartości 0.9.  
Dla 8 z 14 przypadków wartość średnia nie przekracza 0.5, więc znalezione rozwiązania mają co najmniej dwa razy większy koszt.
Dwa najlepsze rozwiązania są dla problemów o rozmiarach 14 i 26.
Widoczne jest jednak, że są to bardzo proste problemy - algorytmy Greedy i Steepest descent mają wartości średnie prawie równe optimum, z niewidocznym odchyleniem standardowym.

Algorytm korzystający z heurystyki w każdym przypadku znajduje lepsze lub równie dobre rozwiązanie, co algorytm średnia z algorytmu losowego.

Średnie rozwiązania dla 200 uruchomień dla algorytmów Steepest descent oraz Greeedy mają bardzo podobne, nierozróżnialne wartości.

\subsubsection{Jakość najlepszego znalezionego rozwiązania}
Poniższy wykres przedstawia jakość najlepszego rozwiązania, znalezionego przez algorytmy, dla poszczególnych instancji.
\begin{center}
\input{./plotQualitySizeMax.tex}
\end{center}
Algorytmy Greedy i Steepest descent zwracają bardzo podobne wyniki.
Jeżeli udało się znaleźć optymalne rozwiązanie, to oba algorytmy je znajdowały.
Dla problemów, gdzie nie udało się znaleźć optymalnego rozwiązania, Steepest Descent osiągał lepsze rezultaty (np. instancje o wielkości 15, 22, 25).
W jednym przypadku (problem chr20a) najlepsze rozwiązanie znalezione przez algorytm Greedy miało lepszą jakość. 

\section{Czas działania w zależności od rozmiaru problemu}
\subsection{Random, Greedy i Steepest descent}
Na wykresie przedstawiony jest średnia i wartość jednego odchylenia standardowego czasu potrzebnego na znalezienie rozwiązania dla 200 uruchomień algorytmów Random, Greedy i Steepest Descent.
Do wyników dopasowaliśmy krzywe wykładnicze.
Pominęliśmy przy tym dla Greedy i Steepest descent problem o rozmiarze 32, które ze względu na bardzo dużą liczbę zer w macierzy zatrzymywały się po kilku krokach.
Dla wszystkich trzech algorytmów istnieje widoczna zależność o charakterze wykładniczym.

Problemy posiadają różne krajobrazy rozwiązań.
Ich charakter może być powodem, dla którego średnie rozwiązania nie układają się dokładnie na przybliżonej charakterystyce, a odchylenia standardowe mają duże wartości.
%TODO Dopisać jakie są parametry dopasowanej krzywej?
%TODO Uzuepłnić rozważaniami o złożoności obliczeniowej?
\begin{center}
\input{./plotTimeSize.tex}
\end{center}

\subsection{Algorytm heurystyczny}
Na podstawie wykresu dla algorytmu heurystycznego możemy również wnioskować o zależności wykładniczej.
Czas obliczenia rozwiązania nie jest dokładnie odwzorowaniem takiej funkcji, bo 
%TODO Dlaczego problem o rozmiarze els19 ma krótszy czas wykonania dla heurystyki niż chr18a?
\begin{center} 
\input{./plotTimeSizeHeuristic.tex}
\end{center}

% Po zwiększeniu czasu Randoma jest na wykresie plotTimeSize
%\input{./plotTimeSizeRandom.tex}

\section{Jakość rozwiązania w zależności od czasu działania}
\begin{center}
\input{./plotQualityTime.tex}
\end{center}
\begin{center}
\input{./plotQualityTimeFull.tex}
\end{center}

\section{Średnia liczba kroków w zależności od rozmiaru problemu}
\begin{center}
\input{./plotStepsSize.tex}
\end{center}

\section{Wnioski}

\end{document}
