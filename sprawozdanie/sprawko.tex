\documentclass[a4paper,10pt]{article}
\usepackage[top=0.85in,right=1in,left=1in,bottom=0.85in]{geometry}
\usepackage{latexsym}
\usepackage{amsfonts}
\usepackage{amssymb}
\usepackage{amsmath}

\usepackage{mathspec}
\usepackage{color}
\usepackage[english,polish]{babel}
\usepackage[plmath]{polski}
\usepackage{xltxtra}
\usepackage{icomma}
\usepackage{textcomp}
\usepackage{multirow}

\usepackage{subfig}
\usepackage{float}
\usepackage{pdfpages}

%\usepackage{minted}

\usepackage{graphicx}

\usepackage{booktabs}
\usepackage[exponent-product = \cdot, per-mode = fraction,%
binary-units, output-decimal-marker = \mbox{,}]{siunitx}

\addto\captionspolish{%
  \renewcommand\tablename{Tabela}
}

\input{_ustawienia.tex}

\begin{document}

\thispagestyle{empty} %bez numeru strony

\begin{center}
{\large{Sprawozdanie z laboratorium:\\
Metaheurystyki i Obliczenia Inspirowane Biologicznie\\}}

\vspace{3ex}

Część I: Algorytmy optymalizacji lokalnej, problem QAP
%Część II: Algorytmy optymalizacji lokalnej i globalnej, problem QAP
%Część III: Eksperyment: ... (prezentację można zrobić w LaTeX - służy do tego klasa "beamer")

\vspace{3ex}
{\footnotesize\today}

\end{center}


\vspace{10ex}

Prowadzący: dr hab.~inż. Maciej Komosiński

\vspace{5ex}

Autorzy:
\begin{tabular}{lllr}
\textbf{Jacek Jankowski} & inf99410 & ISWD & jacek.j.jankowski@gmail.com \\
\textbf{Benedykt Jaworski} & inf99893 & ISWD & sth.jaworski@gmail.com \\
\end{tabular}

\vspace{5ex}

Zajęcia poniedziałkowe, 15:10.

\vspace{35ex}

\noindent Oświadczamy, że niniejsze sprawozdanie zostało przygotowane wyłącznie przez powyższych autorów,
a~wszystkie elementy pochodzące z~innych źródeł zostały odpowiednio zaznaczone i~są cytowane w~bibliografii.  

\newpage



\begin{center}
\input{./plotQualitySizeMean.tex}

\input{./plotQualitySizeMax.tex}

\input{./plotTimeSize.tex}
 
\input{./plotTimeSizeHeuristic.tex}
 
% Po zwiększeniu czasu Randoma jest na wykresie plotTimeSize
%\input{./plotTimeSizeRandom.tex}

\input{./plotQualityTime.tex}

\input{./plotQualityTimeFull.tex}

\input{./plotStepsSize.tex}
\end{center}
\end{document}
